\documentclass[10pt]{article}
\usepackage[utf8]{inputenc}
\usepackage[english]{babel}
\usepackage[margin=0.5in]{geometry} 
\setlength{\parindent}{0pt}

\usepackage{amsmath,amsthm,amssymb,amsfonts,bm}
\usepackage{graphicx}
\usepackage{xcolor}

\definecolor{LightGray}{gray}{0.95}
\definecolor{DarkGray}{gray}{0.1}

\usepackage{minted} % code listing
\usemintedstyle{trac}

\newcommand{\code}[1]{{\small\colorbox{LightGray}{\texttt{#1}}}}


\newcommand{\N}{\mathbb{N}}
\newcommand{\Z}{\mathbb{Z}}

\title{IFT2015 Structure de donnée\\Travail pratique 1}
\author{Jonathan Guymont}

\begin{document}

\maketitle

\section*{Question 1}
(a) Voir \code{sparse.SparseMatrix }.\\

(b) La classe \code{SparseMatrix} encode une matrice dense $A$ de dimension $n\times m$ dans 3 vecteurs (list Python): \code{rowptr}, \code{colind} et \code{data}. Le vecteur \code{colind} contient les indices des colonnes des valeurs non nulles. Les elements de \code{colind} et \code{data} sont données respectivement par les élements en position 2 et 3 des élements de \code{fromiter}
\begin{minted}[bgcolor=LightGray]{python}
colind = [x[1] for x in fromiter]
data = [x[2] for x in fromiter]
\end{minted}
Le vecteur \code{rowptr} est contruit iterativement de la façon suivante:
\begin{minted}[bgcolor=LightGray]{python}
rowptr[0] = 0
for i in np.arange(1, n+1):
    rowptr[i] = rowptr[i-1] + non_zero_counter[i-1]
\end{minted}
où \code{non\_zero\_counter[i-1]} est le nombre d'élements non nuls de la ligne $i-1$ de la matrice $A$. Après que la matrice soit encodée, pour acceder a un élement dont la coordonnée est $(i, j)$, on a besoin de 2 choses: le nombre d'élements non nuls dans les lignes precedentes et les colonnes correspondantes au élement non nulle de la ligne $i$. Le nombre d'elements non nuls dans les ligne precedentes est donne par \code{rowptr[i]}. Les colonnes correspondantes aux élements non nuls de la ligne $i$ sont données par 
\begin{minted}[bgcolor=LightGray]{python}
# colonnes correspondant aux element non nuls de la ligne i
columns = colind[rowptr[i]:rowptr[i+1]]
\end{minted}
puisque les élements dont les indices sont \code{rowptr[i],...,rowptr[i+1]-1} sont les élements de la ligne $i$. Donc si $A[i, j]$ est non nulle, alors $j$ est dans la list \code{columns}. Si $j$ est egale a \code{colind[$k$]} pour \code{rowptr[$i$]} $<= k <$ \code{rowptr[$i$+1]}, alors $A[i,j]$ = \code{data[$k$]}. Sinon, l'élement de la position $(i,j)$ est nul, i.e. $A[i,j] = 0$. On peut maintenant retrouver la matrice dense de la facon suivante:
\begin{minted}[bgcolor=LightGray]{python}
dense = [[self[i, j] for j in range(m)] for i in range(n)]
\end{minted}

(c) La complexité est $O(n)$ ou $n$ est le nombre de ligne de la matrice à encoder. Construire \code{colind} et \code{data} requiert $nnz$ opérations de comlexité $O(1)$ (assigner une variable et l'ajouter a une liste). Construire \code{rowptr} requiert la construction de la liste \code{non\_zero\_counter} fait en $n$ opérations de comlexité $O(1)$ et finalement \code{rowptr} requiert aussi $n$ operations de complexité $O(1)$.


\section*{Question 2} 
Voir la fonction \code{question2()} dans \code{main.py}.

\section*{Question 3}
Soit $A$ une matrice dense de dimension $n\times m$ et supposons qu'on veut l'élement de la position $(i,j)$. Dans le pire cas, le nombre d'elements non nuls de la ligne $i$ est $m-1$ (en supposant que si le nombre d'élements non nuls est $m$, $A[i,j]$ = \code{data[rowptr[i]+j-1]}). L'operarion dominante en terme de complexité de la methode \code{\_\_getitem\_\_($(i,j)$)} est la recherche de l'indice de de la valeur $j$ dans la liste des colonnes des élements non nuls de la ligne $i$. La methode \code{\_find\_index} effectue une recherche binaire pour trouver l'indice de $j$ (voir \code{sparse.SparseMatrix.\_find\_index}). Le nombre d'operations d'une recherche binaire est $t(m-1)=t(\lceil m/2\rceil) + O(1)$. Par un theorem d'algorithmic, si on a la relation $t(n)=lt(n/b)+g(n)$ avec $l=b^0$ et $g(n)\in O(n^0)$, alors $t(n)\in O(n^0\log n)=O(\log n)$.

\section*{Question 4}
(a) Voir \code{sparse.SparseTensor} pour l'interface detaillé. Les vecteurs \code{colind} et \code{data} sont construient de la même façon que pour \code{SparseMatrix}, i.e.
\begin{minted}[bgcolor=LightGray]{python}
colind = [x[2] for x in fromiter] 
data = [x[3] for x in fromiter] 
\end{minted}
Le vecteur \code{rowptr} est construit d'une facon un peu differente. Premièrement, on utilise un indice \textit{flat} pour le tensor: $s = i\cdot n + j$. C'est indice fait en sorte que
\[
	\code{rowptr[$s$]} = \code{rowptr[$s-1$]} + \#\text{elements de la ligne $j$ de l'image $i$}
\]
où $i=1,...,60000$ est l'indice de l'image, $n=28$ est le nombre de lignes dans une image, et $j=1,...,28$ est l'indice de la ligne. Pour recuperer un élement sachant ces coordonnées $(i, j, k)$, on cherche \code{rowptr[$s$]} $\leq p<$ \code{rowptr[$s+1$]}, où $s=i\cdot n+j$, tel que \code{colind[$p$]}$=k$. Si un tel $p$ n'existe pas, on retourn 0, sinon on retourne \code{data[$p$]}.\\

(b) Voir \code{main.question4b()}.\\

(c) Oui, le nombre de ligne nulle dans les images MNIST (i.e. dont tous les pixels sont 0) est 495810 et le nombre total de ligne (nombre d'image $\times$ nombre de lignes par image) est 168000. Aussi, les lignes nulles son souvent regroupées ensemble au debut et a la fin des images. Il serait donc possible de rendre le vecteur \code{rowptr} plus compact. Les vecteurs \code{data} et \code{colind} ne peuvent pas être plus compact.

\section*{Question 5}
(a) Voir \code{main.question5()}. NOTE: La même implémentation qu'à la question 4 est utilisée.\\

(b) L'espace occupé par \code{colind} et \code{data} est $nnz$ pour les deux. L'espace occupé par \code{rowptr} est $l \times n$ où $l$ est la dimension 1 du tensor (e.g. le nombre d'images dans notre example) et $n$ est la dimension 2 du tensor (e.g. le nombre de ligne de pixel dans chaque image). L'espace total occupé est donc $2\cdot nnz + l\cdot n$. L'espace occupé par un tensor dense est $l\cdot n \cdot m$. \\

(c) L'encodage de Yale n'est pas toujours plus compact. Par exemple, si tous les elements du tensor son non nuls, l'espace occupe par l'encodage de Yale est $2\cdot l\cdot n \cdot m + l\cdot n$, ce qui est plus de 2 fois l'espace utilisé par le tensor dense.

\section*{Question 6}
Soit $(i_1,...,i_k, v)$ les coordonnées d'un élement dans un tensor de dimension $k$ et sa valeur $v$. Soit $(n_1,...,n_k)$ les dimensions de chaque coordonnées. Les vecteurs \code{data} et \code{colind} seront
\begin{minted}[bgcolor=LightGray]{python}
data = [x[k] for x in fromiter] 
colind = [x[k-1] for x in fromiter]
\end{minted}
et l'indice $s$ sera donné par $\sum_{j=1}^{k-2} i_j \cdot n_j + i_{k-1}$. Le reste de l'implementation est identique a l'implémentation décrite a la question 4 a).























\end{document}


